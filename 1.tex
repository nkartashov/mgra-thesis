\chapter{Обзор предметной области, существующих решений и постановка проблемы}

\section{Обзор предметной области}
Общеизвестно, что молекула ДНК в процессе эволюции может меняться.
Изменения происходящие в молекуле ДНК можно разделить на две группы:
\begin{enumerate}
  \item Точечные (замены, вставки и удаления на уровне отдельных нуклеотидов)
  \item Структурные (перестройки на уровне отдельных сегментов молекулы ДНК)
\end{enumerate}
На данный момент науке известно множество структурных изменений молекулы ДНК - инверсия, удаление, вставка, слияние/разделение хромосом и другие.
В ходе исследований последних 20 лет ученые пришли к консенсусу, что перестройки не могут происходить в случайных местах генома,
так называемой ``хрупкой'' гипотезе.
На ее основе можно ввести понятие \textit{блоков синтении}.

\begin{define}{Блоки синтении} \\
  В молекуле ДНК существуют консервативные регионы, называемые блоками синтении (синтенными блоками),
  геномные перестройки в которых маловероятны.
\end{define}

Зачастую под блоками синтении понимают гены, потому в данной работе слова ``ген'' и ``блок'' одинаково обозначают ``блок синтении''.
В данной работе геном будет рассматриваться в виде набора хромосом,
где каждая хромосома состоит из набора блоков синтении и молекула ДНК подвержена только структурным изменениям.

Хромосомы в геноме могут быть циклическими и линейными.
Другими словами, каждая хромосома представляется в виде перестановки над синтенными блоками.
Для получения такого представления генома из последовательности нуклеотидов существует целый ряд программных средств,
таких как Sibelia~\cite{minkin2013sibelia}, DRIMM-Synteny~\cite{pham2010drimm}, i-ADHoRe3.0~\cite{proost2012adhore} и другие.

Существует две модели перестановок: знаковые и беззнаковые.
Описание проблем и задач для беззнаковых перестроек приведены не будут, так как данные знания не важны для понимания дальнейшей работы.
Стоит отметить, что использование знаковых перестановок биологически оправданно:
так как молекула ДНК обладает двумя комплиментарными цепями, можно сказать,
что если ген лежит на нити 3' - 5', то он имеет положительный знак, а если на 5' - 3', то отрицательный.

Таким образом, на основе введенных определений можно сформулировать решаемую проблему.
\begin{prob}{Проблема восстановления деревьев} \\
  Для данных в виде набора геномов, в котором каждая хромосома представлена в виде перестановки над блоками синтении,
  восстановить филогенетическое дерево.
\end{prob}

Прежде, чем перейти к обзору существующих решений введем понятие брейкпоинт-графа и способы оценки расстояния между геномами.
Рассмотрим одиночную циклическую хромосому, разбитую на уникальные блоки синтении.
После данного разделения хромосома может быть представлена в виде графа с двумя типами ребер: направленными, обозначающими блоки синтении,
и ненаправленными, обозначающими связи между блоками.
В таком представлении для блока $a$ назовем вершину в которую входит обозначающее его ребро $a_h$, а вершину из которой это ребро исходит - $a_t$.
После такого переименования можно заметить, что даже при удалении направленных ребер информация о них не потеряется.
Таким образом, можно перейти к представлению графа в виде списка смежности вершин,
где вершины считаются смежными, если они соединены ненаправленным ребром.
Данное представление можно обобщить для линейных хромосом, если добавить специальную фиктивную вершину $\infty$, с которой будут связаны первый и последний синтенные блоки в хромосоме.
Далее определим, что для объединения двух хромосом список смежности получается объединением списков смежностей для каждой из них.
Таким образом, мы получили описание того как перейти от набора линейных или циклических хромосом, представленного
в виде списка блоков синтении к графу смежностей этих блоков, представленному списком смежности вершин.

Теперь перейдем к рассмотрению нескольких геномов.
Для начала примем, что геномы определены на одних и тех же блоках синтении и в каждом из геномов они все присутствуют в единичном экземпляре.
Пусть для объединения геномов вышеописанная структура графа определяется как индексированное объединение списков смежности,
где индексами являются любые два несовпадающих цвета из множества цветов.
В дальнейшем вместо слова ``геном'' будет также употребляться ``цвет''.
Смысл предыдущей операции состоит в том, чтобы объединить графы смежности для обоих геномов не теряя информации, к какому геному какое ребро принадлежит.
Структура графа выше имеет название ``брейкпоинт-граф'', также можно определить такую структуру для количества геномов больше двух:
для этого для каждого следующего генома выбирается новый цвет и его ребра добавляются к имеющимся.
Брейкпоинт-граф для множества геномов также называется ``множественный брейкпоинт-граф''~\cite{caprara1999tightness}.
Таким образом, получается структура, хранящая в себе информацию о смежностях блоков сразу во многих геномах определенных на этих блоках.

Имея брейкпоинт-граф для геномов становится возможным ввести оценку филогенетического расстояния между ними.
Для этого введем операцию выполняемую над брейкпоинт-графом, называемую 2-брейк.
\begin{define}{2-брейк (Double Cut and Join, DCJ)} \\
  Назовем 2-брейком следующую операцию: удаление 2 ребер в брейкпоинт-графе и добавление новых двух ребер на ``освободившихся'' вершинах
  несовпадающих с удаленными.
\end{define}
При выполнении 2-брейка над брейкпоинт-графом количество компонент связности в нем может уменьшиться или увеличиться.
\begin{define}{2-брейк расстояние (DCJ-расстояние, $d_{DCJ}$)} \\
  2-брейк расстояние - длина кратчайшей последовательности из 2-брейков приводящей
  исходный брейкпоинт-граф в состояние с наибольшим числом компонент связности.
\end{define}

Данное расстояние может быть расширено для учета для учета различных структурных изменений строения молекулы ДНК.
На практике также используется оценка расстояния, называемая брейкпоинт-расстояние.

\begin{define}{Брейкпоинт-расстояние (BP-расстояние, $d_{BP}$)} \\
  Брейкпоинт-расстояние - расстояние, вычисляемое по формуле $d_{BP} = n - a - \frac{t}{2}$,
  где $n$ - количество генов в каждом геноме,
  $a$ - количество общих связностей для двух геномов, $t$ - количество
  общих связностей, где один конец связности является вершиной $\infty$.
\end{define}

Используя введенные определения расстояний можно перейти к обзору существующих инструментов,
решающих поставленную выше проблему.

\section{Существующие решения}

В решении задачи восстановления филогенетических деревьев есть три основных подхода:
\begin{enumerate}
  \item На основе матрицы расстояний (Distance Methods, DM)
  \item Максимального правдоподобия (Maximum Likelihood, ML)
  \item Максимальной бережливости (Maximum Parsimony, MP)
\end{enumerate}

Далее рассматриваются инструменты представляющие все три подхода.

\subsection{TreeInferer (Ragout) и TIBA}
Оба инструмента восстанавливают деревья с на основе матрицы расстояний.
Восстановление деревьев в данном подходе делится на две части:
поиск попарных расстояний между геномами и восстановление дерева из матрицы известных расстояний
(с возможным пересчетом матрицы на каждом шаге).
И подсчет расстояний и восстановление дерева в данном подходе может осуществляться с использованием различных оценок, что дает подходу гибкость.
TreeInferer, будучи частью сборщика Ragout~\cite{Kolmogorov2014}, может работать с несобранными данными
и использует $d_{BP}$ в связке с Neighbour Joining.
TIBA работает только с собранными геномами, использует оценку $d_{DCJ}$
и либо Neighbour Joining, либо еще один механизм восстановления деревьев на основе матрицы расстояний, FastME~\cite{desper2002fast}.
Стоит отметить, что оба этих инструмента работают только на геномах без вставок и удалений генов
и не могут учитывать информацию об известных поддеревьях.

\subsection{MLWD}
Данный инструмент использует подход максимального правдоподобия.
В данном подходе ключевым моментом является выбор вероятностной модели - математической модели,
которая позволяет оценить правдоподобие филогенетического дерева при условии имеющихся листовых геномов
и далее выбрать то дерево, что имеет наибольшее правдоподобие.
Данный подход делает инструменты с ее использованием труднорасширяемыми,
но взамен позволяет строить модели более точно отражающие эволюционные процессы происходящие в действительности.
MLWD использует данное преимущество и потому поддерживает работу с блоками, полученными из несобранных геномов,
с вставками, удалениями и дупликациями генов, но также как
и прошлые инструменты не может использовать информацию об известных поддеревьях.

\subsection{GAS Phylogeny}
Этот инструмент работает использует подход максимальной бережливости.
Суть данного подхода состоит в построении модели,
в которой каждое эволюционное событие имеет вклад в оценку восстановленного дерева, и дальнейшего нахождения дерева с наименьшей такой оценкой.
Задача поиска филогенетического дерева с наименьшей оценкой принадлежит к классу NP-трудных, потому зачастую при использовании данного подхода
применяются эвристические оценки вместо точных и выбирается возможно неоптимальное, но, тем не менее, имеющее близкую к оптимальной оценку.
В представленном инструменте~\cite{xu2011gasts} используется эвристическая оценка $S_{GASTS}$, позволяющая ему работать быстро и при этом выдавать точный результат.
Данный инструмент не умеет обрабатывать данные, полученные из несобранных геномов, имеющих вставки или удаления и не учитывает информацию об
известных поддеревьях.

\section{Филогенетическая информация в брейкпоинт-графе}

Опишем основания для выбора предлагаемого в данной работе подхода.
Рассмотрим теперь множественный брейкпоинт-граф для геномов и некое неизвестное филогенетическое дерево для них.
В процессе эволюции молекула ДНК подвергалась структурным изменениям при движении от корня филогенетического дерева к листьям.
Имея же на руках брейкпоинт-граф для листовых геномов можно сопоставить операции на нем с поворачиванием вспять истории структурных изменений,
иначе говоря, движением по филогенетическому дереву от листьев к корню:
при выполнении операций 2-брейк, приводящих к увеличению компонент связности, происходит, по сути, обращение структурных изменений,
произошедших на ветвях дерева в процессе эволюции.
Таким образом, в брейкпоинт-графе ``закодирована'' информация о филогенетическом дереве в виде преобразований над корневым геномом,
ребра различных цветов же, содержат такую информацию для каждого из потомков.

Теперь можно описать главную проблему, решаемую в данной работе.

\section{Задача восстановления деревьев}
Главной проблемой в данной работе является проблема восстановления филогенетических деревьев из брейкпоинт-графа.
На пути к ее решению ставятся две задачи:
\begin{enumerate}
  \item Найти способы извлечения информации из брейкпоинт-графа
  \item Найти способы из извлеченной информации построить филогенетическое дерево
\end{enumerate}
