\thispagestyle{empty}
\setcounter{page}{1}

\section*{Реферат}
\noindent
С. \total{page},  рис. \total{figuresnum}, табл. \total{table}.

В данной работе представлены методы извлечения информации из брейкпоинт-графа и восстановления филогенетических деревьев на их основе.
Разработанный алгоритм основан на поиске паттернов в брейкпоинт-графе, их декодирования и дальнейшего восстановления на основе полученной информации.
В отличие от большинства аналогичных методов, он способен работать с данными со вставками и удалениями синтенных блоков, несобранными данными и
использовать информацию об известных поддеревьях.
Также алгоритм позволяет заменять способы извлечения информации из брейкпоинт-графа, что оставляет возможности для его дальнейшего расширения.
Алгоритм реализован как часть программного пакета MGRA2, доступного под лицензией GNU GPL v2.0.


\textbf{Ключевые слова}: брейкпоинт-граф, филогенетические паттерны, восстановление филогенетических деревьев.

\pagebreak
