\intro

Все биологические дисциплины согласны в том, что биологические виды имеют общую историю.
Для выбранной группы организмов данная история может быть представлена в виде филогенетического дерева.
Анализ филогенетических деревьев и предковых геномов является одним из главных инструментов эволюционной биологии.
Восстановление филогенетических деревьев на сегодняшний день может проводиться на основе данных секвенирования.
Новые технологии секвенирования позволяют получать большое количество данных секвенирования относительно дешево и быстро,
потому особенно важно уметь восстанавливать филогенетические деревья на их основе быстро и точно.
Знание филогенетического дерева полезно в ряде приложений, его получение делает возможным восстановление предковых геномов с высокой точностью,
облегчает точную сборку секвенированных геномов, дает понимание о том, как шел процесс эволюции.

На данный момент существует множество программных средств решающих задачу восстановления филогенетических деревьев на основе данных
секвенирования, отличающихся как в подходах к решению задачи, так и в информации используемой для восстановления,
в данной работе представлены два алгоритма, решающие задачу восстановления деревьев из разделений, информация о которых получается из брейкпоинт-графа,
а также рассмотрен метод поиска способов извлечения филогенетической информации из брейкпоинт-графа.
Алгоритмы реализованы как часть программного пакета MGRA2 на языке C++ и доступен под лицензией GNU GPL v2.0.

Диссертация состоит из трёх глав.
В первой главе описывается постановка задачи и дается обзор существующих методов.
Вторая глава посвящена описанию методов получения филогенетической информации и алгоритмов восстановления деревьев на ее основе.
Третья глава содержит сравнение реализованных методов с известными на тестовых данных.
